\documentclass[12pt]{article}

% Pacotes necessários
\usepackage[left=3cm, right=2cm, top=3cm, bottom=2cm]{geometry}
\usepackage{setspace}
\usepackage{fontspec} % Use apenas se estiver usando XeLaTeX ou LuaLaTeX
\usepackage{titlesec}
\usepackage{fancyhdr}

% Configurações de fonte (use apenas se estiver usando XeLaTeX ou LuaLaTeX)
\setmainfont{Times New Roman}

% Configurações de espaçamento
\onehalfspacing

% Configurações de parágrafos
\setlength{\parindent}{1.25cm}

% Configurações de cabeçalhos e rodapés
\pagestyle{fancy}
\fancyhf{}
\fancyfoot[C]{\thepage}

% Configurações de títulos e subtítulos
\titleformat{\section}{\bfseries\centering\Large}{\thesection}{1em}{}
\titleformat{\subsection}{\bfseries\centering\large}{\thesubsection}{1em}{}
\titleformat{\subsubsection}{\bfseries\centering\normalsize}{\thesubsubsection}{1em}{}

% Início do documento
\begin{document}

% Título
\title{Título do Trabalho}
\author{Nome do Autor}
\date{}
\maketitle

% Resumo
\begin{abstract}
  Este é um exemplo de resumo. Descreva brevemente os objetivos, metodologia, resultados e conclusões do trabalho.
\end{abstract}

% Palavras-chave
\textbf{Palavras-chave:} palavra1; palavra2; palavra3.

% Introdução
\section{Introdução}
Este trabalho aborda...

% Revisão de Literatura
\section{Revisão de Literatura}
Conforme Silva (2020)...

% Metodologia
\section{Metodologia}
A pesquisa foi conduzida...

% Resultados e Discussão
\section{Resultados e Discussão}
Os resultados indicam...

% Conclusão
\section{Conclusão}
Em suma, o estudo...

% Referências
\section*{Referências}
\begin{itemize}
  \item SILVA, João. \textit{Título do livro}. 2. ed. São Paulo: Editora, 2020.
  \item PEREIRA, Maria. \textit{Título do artigo}. Revista Científica, São Paulo, v. 10, n. 2, p. 45-56, 2020.
\end{itemize}

\end{document}
