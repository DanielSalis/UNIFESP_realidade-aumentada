\documentclass[conference]{IEEEtran}
\IEEEoverridecommandlockouts
% The preceding line is only needed to identify funding in the first footnote. If that is unneeded, please comment it out.
\usepackage[brazil]{babel}
\usepackage[utf8] {inputenc}
\usepackage{cite}
\usepackage{amsmath,amssymb,amsfonts}
\usepackage{algorithmic}
\usepackage{graphicx}
\usepackage{textcomp}
\usepackage{xcolor}
\def\BibTeX{{\rm B\kern-.05em{\sc i\kern-.025em b}\kern-.08em
T\kern-.1667em\lower.7ex\hbox{E}\kern-.125emX}}
\begin{document}

\title{Desenvolvimento de aplicação web para experimentar mobílias utilizando realidade aumentada\\
}

\author{
  \IEEEauthorblockN{Daniel Amorim Vilela de Salis}
  \IEEEauthorblockA{123.145\\
    \textit{Universidade Federal de São Paulo}\\
    \ São José dos Campos, São Paulo\\
    \ daniel.salis@unifesp.br} \\

}

\maketitle

\begin{abstract}
  Este artigo tem como objetivo demonstrar...
\end{abstract}

\section {Introdução}

\begin{thebibliography}{00}
  \bibitem{b}L. Tremosa. “Beyond AR vs. VR: What is the Difference between AR vs. MR vs. VR vs. XR?” Interaction Design Foundation - IxDF. https://www.interaction-design.org/literature/article/beyond-ar-vs-vr-what-is-the-difference-between-ar-vs-mr-vs-vr-vs-xr (accessed May. 30, 2024).

\end{thebibliography}

\end{document}
